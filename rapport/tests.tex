\documentclass{article}

\usepackage[utf8]{inputenc}
\usepackage[T1]{fontenc}
\usepackage[french]{babel}
\usepackage{lmodern}
\usepackage{fullpage}

\usepackage{pbox}
\usepackage{titlesec}
% \titlespacing*{\chapter}{0pt}{0pt}{0pt}
\usepackage{graphicx}
\usepackage[parfill]{parskip}
\usepackage[colorlinks=true, urlcolor=black, linkcolor=black, citecolor=black]{hyperref}
\usepackage{amsmath}
\usepackage{tabto}

% Listings --------------------------------------------------------------------%
\usepackage{color}
\usepackage[usenames,dvipsnames]{xcolor}
\usepackage{listings}
\usepackage{caption}
\usepackage{wrapfig}

\DeclareCaptionFont{white}{\color{white}}
\DeclareCaptionFormat{listing}{\colorbox[cmyk]{0, 0, 0, 1}{\parbox{\textwidth}{\center{#3}}}\vspace{5pt}}
\captionsetup[lstlisting]{format=listing,labelfont=white,textfont=white, singlelinecheck=false, margin=0pt, font={bf, Large}}
\lstset{
	breaklines=true,
	tabsize=4,
	% basicstyle=\normalsize\ttfamily,
	basicstyle=\large\ttfamily,
	columns=fullflexible,
	showstringspaces=false,
	numberstyle=\tiny,
	numbersep=5pt,
	extendedchars=true,
	showspaces=false,
	xleftmargin=8pt,
	xrightmargin=8pt,
	framexleftmargin=17pt,
	framexrightmargin=5pt,
	framexbottommargin=4pt,
	morekeywords={const, if, then, else, pre, require, const, Set, enum, let, and, in}
}

% use latex commands inside ` ... `
\lstset{escapeinside={\`\`}}
%------------------------------------------------------------------------------%

% Todos -----------------------------------------------------------------------%
\setlength{\marginparwidth}{1.2in}
\let\oldmarginpar\marginpar
\renewcommand\marginpar[1]{\-\oldmarginpar[\raggedleft\footnotesize #1]%
{\raggedright\footnotesize #1}}
\newcommand{\note}[1]{\marginpar{\colorbox{yellow}{\begin{minipage}{\linewidth}\textcolor{black}{#1}\end{minipage}}}}
%------------------------------------------------------------------------------%

\newcommand{\tb}{\textbackslash{}}
\newcommand{\cmd}[1]{\texttt{#1}}
\newcommand{\fleche}{$\rightarrow{}$}
\newcommand{\X}{$\times{}$}
\newcommand{\eqmin}{$\overset{min}{=}$}
\newcommand{\eqdef}{$\overset{def}{=}$}
\newcommand{\supeq}{$\geq$~}
\newcommand{\infeq}{$\leq$~}


\newcommand{\lIN}[1]{{\tb}in \{~#1~\}}
\newcommand{\lAND}{$\land$}
\newcommand{\lOR}{$\lor$}
\newcommand{\lNOT}{$\lnot$}
\newcommand{\lFORALL}{$\forall$}
\newcommand{\lEXISTS}{$\exists$}
\newcommand{\lin}{$\in$}

\renewcommand{\o}[1]{\textcolor{blue}{#1}}
\renewcommand{\c}[1]{\textcolor{red}{#1}}
\newcommand{\eo}[1]{\textcolor{RoyalPurple}{#1}}
\newcommand{\ec}[1]{\textcolor{BrickRed}{#1}}

\newcommand{\obj}[2]{\subsubsection*{\large{\textbf{Objectif {\cmd{#1} (#2)}}}}}
\newenvironment{cas}[1]
{
	\hspace{1em}\textbf{Cas de test} \cmd{#1} :
	\begin{list}{}{}
}{
	\end{list}
}
\newcommand{\pre}[1]{\item \textbf{Préambule :} \cmd{#1}}
\newcommand{\npre}{\item \textbf{Préambule :} aucun}
\newcommand{\ope}[1]{\item \textbf{Contenu :} \cmd{#1}}
\newcommand{\nope}{\item \textbf{Contenu :} aucun}
\newcommand{\ora}[1]{\item \textbf{Oracle :} \cmd{#1}}
\newcommand{\nora}{\item \textbf{Oracle} : aucun}
\newcommand{\oram}[1]{\item \textbf{Oracle :}\\
\begin{lstlisting}
#1
\end{lstlisting}
}

\title{Objectifs de Tests}

\author{}

\date{}

\begin{document}

\maketitle{}

\section{\emph{Stubs} utilisés}

Les objets de test suivant peuvent-être référencés dans les cas de test :

\subsection{Le terrain TER1}

\cmd{TER1 \eqdef{} Terrain::init(5, 3)}

Avec pour blocs :

\begin{verbatim}
.####
.OX.Y
....?
\end{verbatim}

\begin{itemize}
	\item \cmd{X} héro
	\item \cmd{\#} mur
	\item \cmd{.} vide
	\item \cmd{O} rocher
	\item \cmd{Y} diamant
	\item \cmd{?} sortie fermée
\end{itemize}

\subsection{La position POS1}

\cmd{POS1 \eqdef{} Position::init(5, 3, 1, 1)}

\clearpage{}




\section{Bloc}

\subsection{Couverture des invariants}

\obj{Bloc\_invariant1} {Minimisation de isVide}

	\begin{cas}{Bloc\_invariant1}
		\pre{B = init(VIDE, POS1)}
		\nope{}
		\ora{isVide(B)}
	\end{cas}


\obj{Bloc\_invariant2} {Minimisation de isSolide}

	\begin{cas}{Bloc\_invariant2}
		\pre{B = init(ROCHER, POS1)}
		\nope{}
		\ora{isSolide(B)}
	\end{cas}


\obj{Bloc\_invariant3} {Minimisation de isDeplacable}
	\begin{cas}{Bloc\_invariant3}
		\pre{B = init(ROCHER, POS1)}
		\nope{}
		\ora{isDeplacable(B)}
	\end{cas}

\obj{Bloc\_invariant4} {Minimisation de isTombable}
	\begin{cas}{Bloc\_invariant4}
		\pre{B = init(ROCHER, POS1)}
		\nope{}
		\ora{isTombable(B)}
	\end{cas}

\obj{Bloc\_invariant5} {Minimisation de isSortie}
	\begin{cas}{Bloc\_invariant5}
		\pre{B = init(SORTIE\_OUVERTE, POS1)}
		\nope{}
		\ora{isSortie(B)}
	\end{cas}

\obj{Bloc\_invariant6} {Minimisation de isSortieFermee}
	\begin{cas}{Bloc\_invariant6}
		\pre{B = init(SORTIE\_FERMEE, POS1)}
		\nope{}
		\ora{isSortieFermee(B)}
	\end{cas}

\obj{Bloc\_invariant7} {Minimisation de isHero}
	\begin{cas}{Bloc\_invariant7}
		\pre{B = init(HERO, POS1)}
		\nope{}
		\ora{isHero(B)}
	\end{cas}

\obj{Bloc\_invariant8} {Minimisation de isTerre}
	\begin{cas}{Bloc\_invariant8}
		\pre{B = init(TERRE, POS1)}
		\nope{}
		\ora{isTerre(B)}
	\end{cas}


\subsection{Couverture des postconditions}

\obj{Bloc\_init\_post1} {Postcondition de init sur getType}
	\begin{cas}{Bloc\_init\_post1}
		\pre{aucun}
		\ope{B = init(HERO, POS1)}
		\ora{getType(B) = HERO}
	\end{cas}

\obj{Bloc\_init\_post2} {Postcondition de init sur getPosition}
	\begin{cas}{Bloc\_init\_post2}
		\pre{aucun}
		\ope{B = init(HERO, POS1)}
		\ora{getPosition(B) = POS1}
	\end{cas}

\obj{Bloc\_setType\_post1} {Postcondition de setType sur getType}
	\begin{cas}{Bloc\_setType\_post1}
		\pre{B1 = init(HERO, POS1)}
		\ope{B2 = setType(B1, ROCHER)}
		\ora{getType(B2) = ROCHER}
	\end{cas}


\obj{Bloc\_setType\_post2} {Postcondition de setType sur getPosition}
	\begin{cas}{Bloc\_setType\_post2}
		\pre{B1 = init(HERO, POS1)}
		\ope{B2 = setType(B1, ROCHER)}
		\ora{getPosition(B2) = POS1}
	\end{cas}


\subsection{Couverture des transitions}

\obj{Bloc\_setType\_trans} {Transition de setType}
	\begin{cas}{Bloc\_setType\_trans}
		\pre{B1 = init(HERO, POS1)}
		\ope{B2 = setType(B1, ROCHER)}
		\ora{getType(B2) = ROCHER \lAND{} getPosition(B2) = POS1}
	\end{cas}

\end{document}

